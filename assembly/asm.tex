\def\mytitle{\textbf{ASSEMBLY ASSIGNMENT}}
\documentclass[10pt,a4paper]{article} 
\usepackage[a4paper,outer=1.5cm,inner=1.5cm,top=0.6cm,bottom=1.2cm]{geometry} 
\usepackage{graphicx}
\usepackage{amsfonts}
\usepackage{amsmath} 
\usepackage{tabularx} 
\usepackage{titlesec}
\usepackage{listings}
\usepackage[utf8]{inputenc} 
\usepackage{textgreek} 
\usepackage{circuitikz} 
\title{\mytitle} 
\author{Marri Srinath Reddy\\srinathreddymarri@gmail.com\\FWC22139 IITH - Future Wireless Communications}
\date{} 
\begin{document}
\maketitle
%\centerline{\textbf{ASSEMBLY ASSIGNMENT}}
%\centerline{Marri Srinath Reddy}
%\centerline{srinathreddymarri@gmail.com}
%\centerline{FWC22138 IITH - Future Wireless Communications}
\graphicspath{{./Documents}{./figs}}
	\tableofcontents

	\section{Problem}
	(GATE2022-QP-IN)\\
	Q.21 The logic block shown has an output F given by \underline{\phantom{GATE2022IN}}
	\begin{figure}[!h]
		\centering
		\begin{circuitikz}
			\draw
			(0,0) 
			node[label=left:$B$] {}
			-- (7,0)
			(3,0.75) 
			node[not port, anchor=out, scale=0.75] (not1) {}
			(1,0) to[short, *-] (1,0.75) -- (not1.in)
			(0,-2) node[label=left:$A$] {} -- (7,-2)
			(3,-1.25) node[not port, anchor=out, scale=0.75] (not2) {}
			(1,-2) to[short, *-] (1,-1.25) -- (not2.in)
			(3,0.75) -- (7,0.75)
			(3,-1.25) -- (7,-1.25)
			(3.5,-1.25) to[short, *-] (3.5,-3.5)
			(3.5,-3.5) node[nor port, rotate=270, anchor=in 2] (gate1) {}
			(gate1.in 1)
			to[short, -*] ++(0,4.25)
			(5.5,-2) to[short, *-] (5.5,-3.5) 
			(5.5,-3.5) node[nor port, rotate=270, anchor=in 2] (gate2) {}
			(gate2.in 1)
			to[short, -*] ++(0,4.25)
			(4.75,-6.5) node[nor port, rotate=270] (gate3) {}
			(gate3.in 2) |- (gate1.out)
			(gate3.in 1) |- (gate2.out)
			(gate3.out) -- (4.75,-7) node[label=below:$F$] {};
		\end{circuitikz}
		%\includegraphics[width=0.75\columnwidth]{21.jpg}
		\caption{Circuit}
		\label{fig:circuit}
	\end{figure}
	\begin{enumerate}
		\item[(A)] A + B
		\item[(B)] A. $\overline{B}$
		\item[(C)] A + $\overline{B}$
		\item[(D)] $\overline{B}$
	\end{enumerate}
	\section{Components}
	\begin{table}[h]
		\centering
		\begin{tabularx}{0.8\textwidth}{
				| >{\raggedright\arraybackslash}X
				| >{\raggedright\arraybackslash}X
				| >{\raggedright\arraybackslash}X | }
			\hline
			\textbf{Components} & \textbf{Value} & \textbf{Quantity} \\
			\hline
			Breadboard & - & 1 \\
			\hline
			Arduino & Uno & 1 \\
			\hline
			Jumper Wires & - & 4 \\
			\hline
		\end{tabularx}
		\caption{Components}
		\label{table:components}
	\end{table}
	\subsection{Arduino}
	The Arduino Uno has some ground pins, analog input pins A0-A3 and digital pins D1-D13 that can be used for both input as well as output. It also has two power pins that can generate 3.3V and 5V. In this excercise we use input pins, digital pins, GND and 5V.
	\section{Implementation}
	\subsection{Truth table}
	\begin{table}[h]
		\centering
		\begin{tabularx}{0.8\textwidth} {
				| >{\raggedright\arraybackslash}X
				| >{\raggedright\arraybackslash}X
				| >{\raggedright\arraybackslash}X | }
			\hline
			Input & Output \\
			\hline
			0 & 1 \\
			\hline
		        1 & 0 \\
			\hline
		\end{tabularx}
		\caption{Truth Table}
		\label{table:truth_table}
	\end{table}
	\subsection{Boolean Equation}
	By Solving the above problem we obtain as follows :
		\begin{align}
			F &= \overline{AB + \overline{A}B} \\
			F &= \overline{B(A + \overline{A})} \\
			F &= \overline{B}
		\end{align}
	\section{Hardware}
	\begin{enumerate}
		\item connect  one end of a jumper wire to the GND(ground) pin on the Arduino Uno board and other end to the bread's ground rail(-).
                \item connect input to Vcc for logic 1, ground for logic 0.
                \item now execute the circuit using below code.
	\end{enumerate}
	\section{Software}
	Now write the code which is available in below path and upload to the Arduino. \\
	\framebox{https://github.com/SrinathReddyMarri/FWC/blob/master/assembly/asb.asm}
	\section{Conclusion}
	Hence we have implemented the NOR gate for the given circuit using the code above with the help of Arduino.
\end{document}
