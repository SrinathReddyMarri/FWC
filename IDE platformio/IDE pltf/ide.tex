\def\mytitle{IDE ASSIGNMENT}
\def\myauthor{Marri Srinath Reddy}
\def\contact{srinathreddymarri@gmail.com}
\def\mymodule{Future Wireless Communications (FWC)}

\documentclass[journal,12pt,twocolumn]{IEEEtran}
\usepackage{graphicx}
\usepackage{enumitem}
\usepackage{tabularx}
\usepackage{amsmath}
\usepackage{amssymb}
\usepackage{amsfonts}
\usepackage{mathtools}
\usepackage{amsthm}

\title{\mytitle}
\author{\myauthor\hspace{1em}\\\contact\\IITH\hspace{0.5em}-\hspace{0.6em}\mymodule}
\begin{document}
\maketitle
\tableofcontents
\section{\textbf{Question}}
The output F of the digital circuit shown can be written in the form(s)\rule{2cm}{0.4pt}
\begin{figure}[h]
    \centering
    \includegraphics[width=0.4\textwidth]{figs/2x1 mux.jpg}
    \caption{2x1 mux}
    \label{fig:my_label}
\end{figure}
\begin{enumerate}[label=(\Alph*)]
   \item $\overline {A \cdot B}$
   \item $\overline A$+$\overline B$
   \item $\overline{\text A+B}$
   \item $\overline A \cdot \overline B$
\end{enumerate}

\section{\textbf{Answer}}
The above question can be solved by using 2x1 mux and boolean algebra. steps are given below\\
$\rightarrow \overline B \cdot I_0 + B \cdot I_1 = Y$\\
$\rightarrow \overline B\cdot1 + B\cdot0 = Y$\\
$\rightarrow Y = \overline B$\\
$\rightarrow F = \overline A\cdot I_0 + A \cdot I_1$\\
$\rightarrow F = \overline A \cdot 1 + A \cdot \overline B$\\
$\rightarrow F = \overline A + A \cdot \overline B$\\
$\rightarrow F = \overline A + \overline B$\\
from de-morgan's theorem\\
$\rightarrow \overline A + \overline B = \overline A \cdot \overline B$
\section{\centering Truth Table}
\begin{tabularx}{0.45\textwidth}{
    | >{\centering\arraybackslash}X
    | >{\centering\arraybackslash}X
    | >{\centering\arraybackslash}X |
 }\hline
 \textbf{$A$}&\textbf{$B$}&\textbf{$F$}\\
 \hline
 0&0&1\\
 \hline
 0&1&1\\
 \hline
 1&0&1\\
 \hline
 1&1&0\\
 \hline
\end{tabularx}
\section{\textbf{Components}}
    \begin{tabularx}{0.45\textwidth}{
           | >{\centering\arraybackslash}X
           | >{\centering\arraybackslash}X
	   | >{\centering\arraybackslash}X |
           }
           \hline
	   \textbf{Components}&\textbf{Values}&\textbf{Quantity}\\
	   \hline
           Arduino & Uno & 1\\ 
	   \hline 
	   Jumper Wires & M-M & 5\\ 
	   \hline
	   Breadboard & & 1\\
           \hline		
   \end{tabularx} 
 \section{\textbf{Implementation}}
\begin{tabularx}{0.45\textwidth}{			
	| >{\centering\arraybackslash}X		
	| >{\centering\arraybackslash}X				
	| >{\centering\arraybackslash}X|}			
\hline								
    \textbf{Arduino PIN}&\textbf{INPUT}&\textbf{OUTPUT}\\
	\hline
	2&$A$& \\
	\hline
	3&$B$&\\
	\hline
	13&&$F$\\
	\hline
\end{tabularx}

\section{\textbf{Procedure}}
\begin{enumerate}[label={\arabic*}.]
    \item Connect the circuit as per the above table.
    \item Connect inputs to Vcc for Logic 1, ground for Logic 0.
    \item Execute the circuit using the below codes.\\
\end{enumerate}
\begin{tabularx}{0.45\textwidth}{| >{\centering\arraybackslash}X|}
	\hline
	https://github.com/SrinathReddyMarri/FWC\\/blob/master/IDE\%20platformio/mux.cpp\\
	\hline
    \end{tabularx}\\
    \end{document}
