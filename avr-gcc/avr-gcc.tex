\def\mytitle{\textbf{AVR-GCC ASSIGNMENT}}
\documentclass[10pt,a4paper]{article}
\usepackage[a4paper,outer=1.5cm,inner=1.5cm,top=2cm,bottom=1.5cm]{geometry}
\usepackage{listings}
\usepackage{circuitikz}
\usepackage{titlesec}
\usepackage[utf8]{inputenc}
\usepackage{float}
\usepackage{amsmath}
\usepackage{graphicx}
\usepackage{amsfonts}
\usepackage{textgreek}
\usepackage{tabularx}
\title{\mytitle}
\author{Marri Srinath Reddy\\srinathreddymarri@gmail.com\\FWC22139 IITH - Future Wireless Communications}
\date{}
\begin{document}
\maketitle
\graphicspath{{./Documents}{./figs}}
\tableofcontents
	\section{Problem}
	(GATE2019-QP-EE)\\
		Q.36 In the circuit shown below , X and Y are digital inputs, and Z is a digital output. The equivalent circuit is a
		\begin{figure}[h!]
	     	\begin{center}
		\centering
		\begin{circuitikz}[scale=1]
    % 1st not
        \draw (0,0) node[not port,scale=0.8 ] (not) {};
        \draw (3,-0.28) node[and port] (and) {};
        
        \draw (not.in 1) -- ++(-1,0) node[left] {$X$};
        \draw (not.out) -- ++(0,0) coordinate (and.in 1);
        \draw (not.in 1) -- ++(0,-1.7) node[below] {$ $};
    % 1st And
        \draw (and.in 1) -- ++(-1.2,0) node[left] {$ $};
        \draw (and.in 2) -- ++(-3.3,0) node[left] {$Y$};
        \draw (and.out) -- ++(0,0) node[right] {$ $};

        \draw (and.in 2) -- ++(-3,0) coordinate (point);
        \draw (point) -- ++(0,-1.7) -- ++(1,0) node[below] {$ $};
        \draw (and.out) -- ++(0,-0.47) node[below] {$ $};
    %2nd not
        \draw (0,-2.28) node[not port ,scale=0.8] (not) {};
        \draw (not.in 1) -- ++(-0.8,0) node[left] {$ $};
        \draw (not.out) -- ++(0,0) coordinate (and.in 2);
   %2nd ANd
        \draw (3,-2) node[and port] (and) {};
        \draw (and.in 1) -- ++(-2.2,0) node[left] {$ $} ;
        \draw (and.in 2) -- ++(-1.2,0) node[left] {$ $} ;
        \draw (and.out) -- ++(0,0) node[right] {$ $};
          \draw (and.out) -- ++(0,0.7) node[above] {$ $};
    %or
        \draw (6,-1) node[or port] (or) {};
        \draw (or.in 1) -- ++(-1.48,0) node[left] {$ $} ;
        \draw (or.in 2) -- ++(-1.48,0) node[left] {$ $} ;
        \draw (or.out) -- ++(1,0) node[right] {$Z$};
  
     \end{circuitikz}
		\end{center} 
		\end{figure}
			\begin{enumerate}
				\item[(A)] NAND gate
				\item[(B)] NOR gate
				\item[(C)] XOR gate
				\item[(D)] XNOR gate
			\end{enumerate}

	\section{Components}
		\begin{table}[htbp]
		\centering
			\begin{tabularx}{1\textwidth}
			{
				| >{\centering\arraybackslash}X
				| >{\centering\arraybackslash}X
				| >{\centering\arraybackslash}X |}
			\hline
			{\bf Components} & {\bf Value} & {\bf Quantity} \\
			\hline
			Arduino & Uno & 1\\
			\hline
			BreadBoard & &  1 \\
			\hline
			Jumper Wires & & 4 \\
			\hline
		\end{tabularx}
			\caption{Components}
			\label{table=Components}
		\end{table}

	\section{Implementation}
				\subsection{Boolean Expression}
 				By solving above expression we get :
				\begin{align}
					z =& \bar{x} . y +x.\bar{y}\\
					z =& \bar{x}y+x\bar{y}  
				\end{align} 
			\subsection{Truth Table}

                        \begin{table}[htbp] 
				\centering  
				\begin{tabularx}{0.5\textwidth}
                        {  | >{\centering\arraybackslash}X
                           | >{\centering\arraybackslash}X
                           | >{\centering\arraybackslash}X |}
                         \hline
                         {\bf A} & {\bf B} & {\bf OUT} \\
                       \hline 
			0 & 0 & 0\\   
			\hline  
			0 & 1 & 1 \\
                       \hline                                                                                                             
	         	1 & 0 & 1 \\  
		        \hline   
			1 & 1 & 0 \\
			\hline
			\end{tabularx}
                        \caption{Truth Table}
				\label{table=truth}
			\end{table}
     \section{Hardware}
	     \begin{enumerate}
		     \item Make the connections between the arduino and Breadboard as shown in Table3.
			     \begin{table}[h]
				     \centering
				     \begin{tabularx}{0.5\textwidth}
					     {
						     | >{\centering\arraybackslash}X
						     | >{\centering\arraybackslash}X
						     | >{\centering\arraybackslash}X |}
						     \hline
						     {\bf Arduino} & 5.0v & GND \\
						     \hline
						     {\bf Breadboard} & +ve & -ve  \\
						     \hline
				     \end{tabularx}
				     \caption{\label{Table-3}Connections}
			     \end{table}
		\item Connect one end of a jumper wire to the GND(ground) pin on the Arduino Uno board and other end to the breadboard’s ground rail(-).
                \item Connect one terminal of jumper wire (Input A) to the input pins on the Arduino(e.g., pin2) and other terminal to the positive rail(+) on the breadboard.
		\item Connect one end of another jumper wire (Input B) to the input pin of Arduino(e.g., pin3) and other end to the positive rail(+) on the breadboard.
		\item Enable the power supply to breadboard from arduino by connecting one end of jumper wire to the power pin of Arduino(5V) and other end to the positive rail(+) on the breadboard.
		\item Change the connections of input pins on the breadboard for differrent outputs. 
	     \end{enumerate}
     \section{Software}
	     Now write the code which is available in below path and upload to the Arduino.\\
	     \framebox{https://github.com/SrinathReddyMarri/FWC/blob/master/avr-gcc/main.c}
	     \section{Conclusion}
	     Hence, the given circuit is equivalent to XOR gate. 
\end{document}
